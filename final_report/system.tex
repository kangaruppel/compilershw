\section{System Overview}

The overarching algorithm is fixed-point - iterating through the functions, summarizing new information about the I/O flow, and iterating through functions again, until there is no new information to be gleaned. Psuedo-code is presented in Algorithm \ref{alg:fixedp}

\begin{algorithm}
\begin{algorithmic}
\Function{searchFunctions}{$Module M$} 
	\ForAll{Function func in M} 
		\State $workinglist \gets pair(func, NULL)$
	\EndFor
	
	\While{!workinglist.empty} 
		\State $currPair \gets workinglist.next$
		\State $sources \gets \Call{findTainted}{currPair}$ 
		\State $sinks \gets \Call{summarize}{sources}$
		\ForAll{taintedPair in sinks} 
			\State workinglist.add(taintedPair)
		\EndFor
	\EndWhile
\EndFunction
\\
\Function{findTainted}{func, ioVal} 
	\State $tainted \gets \Call{searchAnnotation}{func}$ 
	\State $visitlist \gets tainted$ 
	\While{!visitlist.empty}
		\State $current \gets visitlist.next$
		\State $result \gets \Call{traverseDependence}{current}$ 
		\ForAll{Instructions in result} 
			\If{instruction == BranchInst}
				\State \Call{check}{instruction, visitlist}
			\Else 
				\State Sources.add(instruction) 
			\EndIf
			
		\EndFor
	\EndWhile
	\State \Return{Sources}
\EndFunction
\\
\Function{traverseDependence}{tainted} 
	\ForAll{data dependencies of tainted} 
		\If{instruction == (BranchInst or Source)}
			\State result.add(instruction)
		\EndIf
		
	\EndFor
	
	\State \Return{result}
\EndFunction
\\
\Function{check}{branch, visitlist}
\ForAll{conditional target blocks on branch} 
	\ForAll{Instructions in block} 
		\If{instruction == store}
			\State bugReport[branch].add(instruction)
			\State visitList.add(instruction)
		\EndIf
	\EndFor
\EndFor
\EndFunction
\end{algorithmic}
\caption{Iterative, fixed point taint tracking. While the algorithm is not truly interproeedural, it tracks the data and control dependencies with the local function and summarizes how the taint flows out of the function. The summary information is used to add functions to the working list, until the entire I/O tainted flow has been explored}
\label{alg:fixedp}
\end{algorithm}
	
When the pass starts iterating, it doesn't know where the I/O enters the program, so all the functions are put in the working list, with a NULL I/O parameter--the I/O call is found internally, based on annotated information in the source code. Once the I/O source is found, the pass calls a function that traverses the data flow chain of the I/O instruction within the local function. In the chain, there are some instructions of interest: branches, which can lead to the bug of mismatched non-volatile write sets, and instructions that result in I/O tainted values flowing out of the current function that we're examining (an interprocedural flow \bf{source} ). 
	
Traversing the data flow primarily uses LLVM's built-in def-use chains. From the definition of the tainted variable, we follow the uses in a depth first search, adding on any new store instructions to the visit stack. 
From the traversal function call, we return the list of interesting instructions found to $findTainted$. If the instruction was a branch, the pass calls the $check$ function to examine the non-volatile write sets of the branch target blocks to see if there is the possibility for the bug. Additionally, any variables written to on the tainted branch are control-dependent on the I/O variable, so the pass adds any globals to the interprocedural source list and continues the local traversal with the newly tainted control dependent variables.

Once the pass finishes traversing the tainted flow within the local function, it returns back to $searchFunction$ with the list of tainted sources of interprocedural flow.
	
Values can leave the local function by: 
\begin{itemize}
\item{being returned from the function} 
\item{being passed into a global variable} 
\item{being passed as a parameter into another function}
\item{being stored into a call-by-reference parameter (which may be global itself}
\end{itemize}	
	
If the tainted value was returned from the function F, we add all the callers of F to the stack, with the value where the return was stored as the tainted I/O value. (When adding new functions to the examination stack based on summary information, we pass in the tainted value as a parameter instead of searching for the annotations, but then search the chain in the same way. )
	
Likewise if the tainted value was passed into a global, we find the uses of the global, the parent Function of those uses, and add those to the stack, with the global as the tainted value. If the I/O was used as a function parameter, we find the callee function and the local version of the argument. 
	
If I/O was stored into a pass-by-reference parameter, we find all the caller functions and their local value that the argument aliases to. (If the reference para refers to a global, we also add that global to the summary information.)
	
We add all these new function/tainted value pairs to the stack, and continue to iterate. (We do keep track of what pairs we have already examined, to prevent infinite looping). Eventually, there will be no new information being returned from the traversal function, and we have examined all the functions to which the tainted I/O has spread. 
	
\subsection{NV Write-set algorithm}

To help with handling arbitrary complexity of control flow within the branches off of io, we precompute the may write sets of each function before hand, using a bottom up examination of the call graph. 
	
What stores we decide to add the write sets does effect what possible bugs will be reported. For instance, one complexity that must be considered is how to handle stores on branches. Say we have a simple program with two nonvolatile variables, nv0 and nv1 that branches on an I/O tainted variable. On one side of the I/O branch we write to nv0 and nv1. On the other side, we write to nv0, but then we branch again. On one side of the branch we write to nv1, but on the other side we don't. Thus, in this example, one execution could have the I/O branch writing to the same nv set on both sides, and another execution doesn't.
	
There are a few ways to deal with these branches: don't add any stores within the branch basic blocks to the write set, add all the stores to the write set (what we do right now), or make two versions of the write set, adding a different side of the branch to each one. This last option is perhaps most attractive for accuracy, but it is difficult to handle arbitrary complexity, and has a much higher storage overhead.
	
Another problem with the first two approaches is that if we adjust the above example a little, we can have an I/O branch where both sides write to nv0, and on the second branch, one side also writes to nv1. In this case, adding all the stores with the branches detects more bug possibilities, whereas ignoring the branch under reports. So neither of the simpler solutions can be said to definitively under or over report bugs. 
	
\subsection{Filtering Pass}
//TODO

\subsection{High-level, English explanation of what the tool detects}
\begin{itemize}
\item{The final bug is always the disparate nv write sets after a branch off of an I/O tainted value}
\item{We track the I/O through explicit (data) and implicit (control) flow}
\item{Tainted I/O can leave a function through function calls, return values, pass by ref, and globals}
\item{Initial I/O into the prog has to be annotated (not good or bad, just a thing)}
\end{itemize}	
\subsection{Possible limitations of the tool}
\begin{itemize}
\item{Precomputed write sets}  
\item{Overhead?}
\item{How complex can the src code get? -- we should prob impose a limitation on branching levels or function depth}
\item{Are there any other suspicious bug patterns? i.e., completeness}
\item{Over or under report bugs? Right now tending towards conservatism}
\end{itemize}	