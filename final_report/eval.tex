\section{Evaluation}
%-MiBench
%-Tirtos
%-benchmarks from prior (and upcoming) intermittent systems paper evals
%
%-Process
% -remove device specific pragmas from code
% -specify IO functions
% -compile with clang
% -run tool
% -Analyze output warnings, check if the bug would actually manifest
%
To evaluate the pass and demonstrate the presence of I/O dependent branch
bugs in real code, we ran the pass on several suites of code for embedded
devices. The three suites were MiBench~\cite{mibench}, TI-RTOS driver
code~\cite{tirtos}, and modified benchmarks written by the ABSTRACT research
group for testing intermittent runtime systems on energy harvesting
devices~\cite{alpaca,chain,dino,capybara}.  We found that the presence of the
bug varies depending on how the code processes I/O and whether the developers
were aware of the intermittent execution model.

To run the pass, we had to first ensure that the source code could be compiled
with clang so that LLVM-IR could be emitted. To run MiBench and TI-RTOS code we
had to change [insert stuff changed for MiBench and tirtos here]. The ABSTRACT
benchmarks were all written to target MSP430 microcontrollers using TI's
proprietary MSPGCC compiler, so MSP430 specific macros had to be removed from
the benchmark code and the linked runtime.
%
Next, we registered the I/O functions in each of the benchmarks so that the pass
could track their updates. At that point the benchmarks are compiled with
clang so that debug information is included in the LLVM-IR that is produced and
handed to the pass described in Section~\ref{sec:system}. Then we analyzed any
bug reports produced by the pass to determine the severity of the problem.

Initially, the evaluation focused on MiBench~\cite{mibench} which is a benchmark
suite for embedded systems. The problem, however, is the benchmarks tended to
simulate sensor input in a way that does not mimic energy harvesting devices. In
MiBench, applications tended to operate on a stored buffer of data rather than
periodically processing new samples produced by a sensor. Since the MiBench
applications' interaction with sensors is neither frequent nor critical to
control flow, the MiBench codes produced few I/O dependent branch bugs.

The next set of benchmarks came from TI-RTOS driver code~\cite{tirtos}. These
benchmarks contain drivers for sensors that can be called through the TI-RTOS
API. The drivers are written assuming the device is continuously powered or uses
managed sleep cycles to reduce power consumption.
%Are the ti-rtos drivers NV-RAM agnostic?
The benchmarks in TI-RTOS contained several critical I/O dependent branch bugs.
This outcome is not unexpected- the TI-RTOS suite is intended for use by
real-time, event driver applications that react to sensor input, so control flow
in the drivers is often determined by sensor input. Further, the code deals with
low-level manipulations of sensors without knowledge of power failures, so
simple bugs that would be masked under continous execution are likely and become
critical under intermitent execution.

