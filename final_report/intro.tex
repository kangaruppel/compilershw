\section{Introduction}

Batteryless, energy harvesting devices are positioned to deliver an exciting new
class of applications enabled by maintenance free sensing and computing.
Batteryless devices harvest energy from their environments and store it in a
small buffer such as a capacitor rather than charging a battery.  The buffered
energy is used to operate a microprocessor as well as an array of peripherals
that gather and communicate data. As a result, these devices operate only
intermittently as energy is available.
%
Once all of the energy in the buffer has been used, the device powers down and
execution is terminated. Multiple systems have been developed to support long
running computations by continuing the execution across reboots. The
intermittent runtime systems developed in prior work use careful book-keeping of
program state to correctly perform the continuations  without custom hardware
~\cite{chain,dino,mementos,alpaca,mayfly,ratchet}.

When an execution is interrupted by a power failure, intermittent runtime
systems restore the most recent consistent state from a
checkpoint~\cite{mementos,ratchet}, or task boundary~\cite{chain,alpaca,dino} and
re-execute the segment of code that was interrupted.
%
All of these systems allow intermittently executed programs to make progress and
prevent persistent state from being corrupted by re-execution, however, their
guarantees in terms of sensor input are limited. Specifically, if \emph{control
flow} decisions are made based on non-idempotent operations, such as reading
input from a sensor, these systems allow updates from multiple branches to
persist.
%
This is a problem because re-executing the code and taking different branches on
each execution may result in writes to different variables, leaving the
previously written memory in an inconsistent state.
%
Bugs that can result from the partially written state are difficult for
programmers to diagnose because they require the programmer to reason about how
partial execution of different branches of control flow will affect the future
state of the program.
%

Instead, our project uses a compiler pass to statically identify writes to
non-volatile memory that may leave memory in an inconsistent state if repeated
reads from the sensor produce different values. The pass first detects data
dependences between variables in the program and I/O operations. Variables with
data dependences on I/O are considered tainted. Then the pass detects control
flow that is dependent on tainted values. The pass searches for writes to
non-volatile memory in each of the basic blocks along the different control flow
paths. After filtering the writes to reduce the number of false positives, the
pass reports each of the remaining writes as potential bugs.

%In particular, our project focuses on refining the pass to
%reduce the number of false positives the compiler generates by running the pass
%on benchmarks used in intermittently powered devices. Initially, we intended to
%use the MiBench suite of benchmarks for embedded devices, but we found that
%MiBench applications primarily benchmark the computation performed by an
%embedded device, not the device's handling of I/O. In the benchmarks we
%investigated, sensor data are often represented as buffer of data that is
%loaded once at the beginning of the application as opposed to pulled in
%repeatedly as in a typical sensing application.


